\chapter{Production considerations}
\label{chapter:production}

In this chapter I will outline some key considerations regarding running the SaaS system in a production environment, and its differences to the development environment.

Starting from the frontend the React application needs to be built and bundled using Vite and a production ready container configuration needs to be defined.
On the Laravel backend's side, Sail is strictly intended as a development solution and the provided Docker container is quite heavy.
A lighter base (e.g., Alpine Linux) should be used and configured with production ready environment options, and live services.
A production ready PHP server should also be employed, such as FrankenPHP.
Moreover, Mailpit should be replaced with a proper SMTP server, and MinIO needs to be replaced with either its own enterprise solution or another provider's S3 compatible object storage solution.
A database solution that is easy to scale and is enterprise ready is heavily recommended as well.
The third proxy app's Docker configuration should also be edited and a production grade Kubernetes cluster should be used.

Many cloud-based services exist for the above steps that provide out of the box configuration or detailed help.
Of course the whole SaaS or parts of it could be self-hosted on an on-site infrastructure too if the resources are available.
Nowadays, however it is both cheaper and more convenient to rely on cloud providers, so I am going to focus on this solution.
Most notable out of the providers is the big three, namely Azure, AWS, and GCP, which are all well regarded options.
They also provide database, object storage and mailing services that can be used out of the box and scaled if needed.
However, one should consider backups or replicas in other services as well.
If the SaaS uses one provider exclusively that can prove as a single point of failure, should the provider go offline for a while due to an error in the service.

I suggest the Kubernetes cluster to be based on a cloud provider's solution even if one chooses the main app to be on-site.
Such concrete options are Azure AKS or AWS EKS, as they have vast amounts of compute resources available.
In production the revision history limit and replica set configuration's may need to be reviewed and edited accordingly.
Horizontal autoscaling is beneficial to set up as well to handle increased loads to deployed websites.
This can be considered for parts of or the whole of the main application as well.
Many providers also have easy to set up and use options for monitoring the cluster, gathering usage and data analytics, and reporting problems or errors, which I highly recommended looking into.
Lastly a one needs to acquire a domain needs to be used for the main application and the deployed sites.

For continuous integration and delivery of the SaaS system a staging environment should be deployed as well for testing the new version. In a cloud-based production environment the available resources should make this feasible.
Additionally, a blue-green deployment strategy, where two identical production versions of the system, the new and old run simultaneously could be used.
This would allow for switching between the new and old version of the SaaS with zero downtime on updated.
For the individual deployed sites since users can choose when to update with their new designs this is already solved.

I believe the above consideration should help one get started in deploying the developed SaaS into production and help consider various scenarios and use cases.
Of course every situation is unique, so a solution fitting the requirements should be used that fits those needs.
Not only that, as the deployed system evolves, changes, and grows the deployment and operational strategy should be fitted to the emerging problems and needs.