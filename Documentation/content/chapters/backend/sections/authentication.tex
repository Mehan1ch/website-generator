\section{Authentication}
Moving on from the ORM features, next I will describe the implementation of the authentication capabilities of the backend.
The combination of two packages maintained by the Laravel team provide the functionality, the first being Laravel Sanctum, while the second is Laravel Fortify.

Laravel Sanctum provides API token based, SPA session based and Mobile token based authentication methods and provides CORS (Cross-Origin Resource Sharing) and CSRF (Cross-Site Request Forgery) protections for better security.
Out of these three I only used the SPA session based authentication using cookies for the frontend.
To begin with the \texttt{SANCTUM\_STATEFUL\_DOMAINS} environment variable has to be set with the frontend and proxy URLs.
Next, in the application bootstrapper the stateful API global middleware has to be applied.
Lastly the CORS configuration should be published, and the allowed paths have to be set.

In contrast, Laravel Fortify provides a frontend agnostic implementation of authentication features for the backend.
In other words it implements the routes and controllers for login, registration, password reset and confirmation, email verification and two-factor authentication if needed.
Each of these features can be enabled or disabled in its own configuration file.
Additionally fortify can provide Blade templating based views for each of these features if needed however since I will use an SPA frontend I disabled them.
I also opted-out from the two-factor authentication capabilities, the rest however are fully supported by the developed application suite.

One small caveat with using a separate React frontend is that the default password reset and email verification links that are generated by the backend navigate to itself instead of the frontend.
I fixed this by overriding the link generator methods in the application service provider.
To finalize the authentication implementation the user model has to be extended from the \texttt{Authenticable} class, and has to implement the \texttt{MustVerifyEmail} interface.