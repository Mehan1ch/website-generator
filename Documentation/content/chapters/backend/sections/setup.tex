\section{Setup}
I implemented the backend primarily using the Laravel PHP web framework as stated beforehand.
As the first step, the application has to be installed and set up.
This can be done with Laravel's own installer, where many application templates can be selected.
A React template is available, however, that uses Inertia.JS by default, which is a package that enables the frontend to be server-side routed.
I wished to opt out of this and develop a traditional API based SPA, so I decided to use the minimal setup.
So Laravel will serve as an API backend, and the view layer of the MVC architecture will be entirely handled by the frontend.

After installation, I decided to install the Laravel Sail package, which provides a preconfigured development Docker Compose setup, and also provides options to install additional components into our environment.
The extra containers in our case will be the MySQL database, the Mailpit SMTP test server, and the MinIO S3 compatible object storage.
However, the Laravel creators published Sail as a package, which, by default would need a local PHP and Composer (PHP's package manager) installation.
This would kind of defeat its purpose.
Luckily, this can be avoided by using a temporary PHP container to quickly install initial dependencies and then use the Sail application for further required packages.

Afterwards, the provided \texttt{.env.example} file needs to be copied to \texttt{.env} and the appropriate environment variables need to be set, including the mail server, database, and object storage connections.
In the MinIO object storage, a bucket to store files in must be created and set to public, so that the links generated to the files can be accessed by the other applications.
Finally, the application key must be initialized using the Artisan CLI and the default database migrations have to be run.