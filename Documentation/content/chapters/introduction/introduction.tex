%----------------------------------------------------------------------------
\chapter{\bevezetes}
%----------------------------------------------------------------------------

Websites and web applications are part of everyday life: businesses, clubs, and people use them to share information, advertise, or show their work.

For many small organizations and individuals, for example someone who wants a simple resume site or a small local shop advertising its services, having a website should be quick, affordable, and easy to manage.
In practice, creating and keeping a website running often requires time, money and technical knowledge.
Hiring a developer or an agency can be expensive, building and maintaining a site yourself can be confusing, time-consuming and requires deep technical knowledge.

A software-as-a-service (SaaS) platform for websites tries to solve this problem by offering an all-in-one service.
Easy tools to build the site, hosting, so the site is always available, and simple ways to update content without expecting professional skills.

There are many existing solutions already on the market tailored to complex needs.
However, most of them have a steep learning curve, and provider either no free tier or a very limited one.
As a site gets complex, a subscription in these services will be unavoidable.

This thesis presents a free, lightweight SaaS designed for simple use cases, websites made of static pages that display essential information with minimal setup.

\section{Overview}

In Chapter \ref{chapter:literature} I review background material needed to understand the concepts used later.
This includes the idea of software-as-a-service (SaaS), an overview of the Laravel backend framework, a summary of React and client-side technologies, and an introduction to Kubernetes and key deployment concepts.

Chapter \ref{chapter:architecture} defines the functional and non-functional requirements of the platform and presents the system-level architecture.
I describe the purpose and general behaviour of each component and how they work together.

Chapter \ref{chapter:backend} begins the implementation, focusing on the backend: the data model, core services, security and identity handling, integrations with external services, and the logic that provides the platform's main features.

Continuing the development, Chapter \ref{chapter:frontend} covers the frontend implementation, presenting the user interface, components, and typical user flows for site creation and management.

Finishing the main development areas, Chapter \ref{chapter:proxy} addresses deployment and operations by describing a lightweight proxy application used for deploying and serving user websites.

Moving on, Chapter \ref{chapter:testing} provides the testing strategy and results to demonstrate correct behaviour, and offers a full end-to-end example showing how an end user would use the SaaS.

Chapter \ref{chapter:production} discusses differences to consider when moving to production and practical considerations for operating the service.

Finally, Chapter \ref{chapter:summary} summarizes the work and suggests directions for future improvements.