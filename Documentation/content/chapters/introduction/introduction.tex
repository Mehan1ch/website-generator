%----------------------------------------------------------------------------
\chapter{\bevezetes}
%----------------------------------------------------------------------------

Websites and web applications are part of everyday life: businesses, clubs, and people use them to share information, advertise, or show their work.

For many small organizations and individuals, for example someone who wants a simple resume site or a small local shop advertising its services, having a website should be quick, affordable, and easy to manage.
In practice, creating and keeping a website running often requires time, money and technical knowledge.
Hiring a developer or an agency can be expensive, building and maintaining a site yourself can be confusing, time-consuming and requires deep technical knowledge.

A software-as-a-service (SaaS) platform for websites tries to solve this problem by offering an all-in-one service: easy tools to build the site, hosting, so the site is always available, and simple ways to update content without requiring professional skills.

Many such platforms exist on the market, including popular solutions like Squarespace, Wix, GoDaddy, and Shopify.
However, most of them are tailored to complex needs and come with a steep learning curve.
They typically provide either no free tier or a very limited one, and as a site grows more complex, a paid subscription becomes unavoidable.

Consider Wix, one of the most popular website builders with millions of users worldwide.
It offers an extensive drag-and-drop editor with hundreds of templates, e-commerce capabilities, advanced SEO (Search Engine Optimization) tools, and marketing integrations.
While powerful, this breadth of features comes with complexity: users must navigate through numerous menus, settings, and configuration options.
Wix's free tier displays prominent Wix branding and provides limited storage and bandwidth, making it unsuitable for professional use.
To remove branding and access essential features, users must subscribe to paid plans, with full-featured e-commerce plans costing even more.

Similarly, Shopify targets e-commerce specifically, offering inventory management, payment processing, shipping integrations, and detailed analytics.
Its comprehensive feature set makes it ideal for serious online retailers, but also means a steeper learning curve as users must understand product variants, shipping zones, tax settings, and more.
Shopify requires a minimum subscription with no free tier, and transaction fees apply unless using Shopify Payments.

While these platforms excel at their intended use cases, complex websites and professional e-commerce, they are overcomplicated and cost-prohibitive for users who simply need a basic informational website.
Someone wanting to display a resume, showcase a portfolio, or create a simple landing page for a local service does not need inventory management, payment gateways, or advanced marketing automation.

This thesis presents a free, lightweight SaaS platform designed specifically for these simpler use cases: websites made of static pages that display essential information.
By focusing on basic functionality rather than comprehensive features, the platform offers a significantly reduced learning curve.
Users can create and publish a site in minutes without navigating complex settings or paying subscription fees.
The trade-off is intentional: while the platform cannot match the advanced capabilities of Wix or Shopify, it provides exactly what is needed for straightforward informational sites, nothing more, nothing less.

\section{Overview}

In Chapter \ref{chapter:literature}, I review background material needed to understand the concepts I used later.
This includes the idea of software-as-a-service (SaaS), an overview of the Laravel backend framework, a summary of React and client-side technologies, and an introduction to Kubernetes and key deployment concepts.

Chapter \ref{chapter:architecture} defines the functional and non-functional requirements of the platform and presents the system-level architecture.
I describe the purpose and general behaviour of each component and how they work together.

Chapter \ref{chapter:backend} begins the implementation, focusing on the backend: the data model, core services, security and identity handling, integrations with external services, and the logic that provides the platform's main features.

Continuing the development, Chapter \ref{chapter:frontend} covers the frontend implementation, presenting the user interface, components, and typical user flows for site creation and management.

Finishing the main development areas, Chapter \ref{chapter:proxy} addresses deployment and operations by describing the lightweight proxy application I used for deploying and serving user websites.

Moving on, Chapter \ref{chapter:testing} provides the testing strategy and results to demonstrate and validate correct behaviour, and offers a full end-to-end example showing how an end user would use the SaaS platform.

Chapter \ref{chapter:production} discusses differences to consider when moving to production and practical considerations for operating the service.

Finally, Chapter \ref{chapter:summary} summarizes my work and suggests directions for future improvements.