\section{Backend}
\subsection{Setup}
The backend is written primarily using the Laravel PHP web framework as stated beforehand.
As the first step the application has to be installed and set up.
This can be done with Laravel's own installer, where many application templates can be selected.
A React template is available, however that uses Inertia.JS by default which is a package that enables the frontend to be server side routed a minimal API backend template.
So Laravel will serve as an API backend and the view layer of the MVC architecture will be entirely handled by the frontend.

After installation, I decided to install the Laravel Sail package which provides a preconfigured development Docker Compose setup, and also provides options to install additional components into our environment.
The extra containers in our case will be the MySQL database, the Mailpit SMTP test server and the MinIO S3 compatible object storage.
Sail is however published as a package which by default would need a local PHP and Composer (PHP's package manager) installation.
This would kind of defeat its purpose.
Luckily this can be avoided by using a temporary PHP container to quickly install initial dependencies and then use the Sail application for further required packages.

Afterwards the provided \texttt{.env.example} file needs to be copied to \texttt{.env} and the appropriate environment variables need to be set, including the mail server, database and object storage connections.
Finally, the application key must be initialized using the Artisan CLI and the default database migrations have to be run.

\subsection{Data Layer}

Let me start building up the backend's feature set from the bottom, starting with the parts of the Eloquent ORM.
The database schema of the backend including the most important tables can be seen on Figure \ref{fig:database}.

\textit{Note:} Laravel has some other default tables, related to built-in features such as jobs and caching that I omitted from the diagram for better clarity as they will be unused in the application.

\begin{figure}[H]
  \centering
  \includegraphics[width=\linewidth]{./figures/database.png}
  \caption{Database schema}
  \label{fig:database}
\end{figure}

The 'users' table is auto generated, the only change I made is regarding the ID (Identifier) field.
In all tables and models I opted to switch from the default integer ID to UUIDv7 (Universally Unique Identifiers version 7) which are time-based sortable universally unique identifiers composed of 32 hexadecimal digits.

The role related tables are added by the Spatie's Laravel Permission package, which I will talk about in more detail in the authorization Subsection \ref{subsection:authorization}.

On the other hand the media table is added by

\subsection{Authentication}
\subsection{Authorization}
\label{subsection:authorization}
\subsection{Core CRUD}
\subsection{Deployment Logic}
%TODO: Sequence diagram for deployment logic?
\subsection{Admin Panel}