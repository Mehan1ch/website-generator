\section{Laravel}

In this section I will be reviewing Laravel, and its most important features regarding the thesis.
Laravel is a popular PHP web framework created by Taylor Otwell and first released in June 2011 \cite{stauffer2019laravel}.
The framework was built upon Symfony which is another PHP web application framework first and foremost, but also contains a wide variety of PHP tools and libraries.
It was also heavily inspired by Ruby on Rails.

``Laravel is, at its core, about equipping and enabling developers.
Its goal is to provide clear, simple, and beautiful code and features that help developers quickly learn, start, and develop, and write code that's simple, clear, and lasting \cite{stauffer2019laravel}.''
The tools and ecosystem of the framework allow developers to write more scalable and maintainable code.

Laravel uses the Model-View-Controller (MVC) architecture.

Models represent logical entities or resources in the application and is implemented by the Eloquent ORM (Object-Relational Mapping) of Laravel.
Controllers provide methods to handle incoming requests and return an appropriate response or serve a view.
Views display the aforementioned response from a controller method.
Laravel provides the built-in Blade templating engine to return HTML views to the users \cite{subecz2021web}.

An example diagram visualizing this architecture can be seen on Figure \ref{fig:mvc}.

\begin{figure}[H]
  \centering
  \includegraphics[width=\linewidth]{./figures/mvc.png}
  \caption{Model-View-Controller architecture}
  \label{fig:mvc}
\end{figure}

\subsection{Request Lifecycle}

Aside from understanding the general architecture of Laravel getting a grasp of its inner workings through the request lifecycle allows one to better understand on a high-level how the framework truly functions.
As all web applications, Laravel is also served from a web server commonly Apache, Nginx or nowadays popularly FrankenPHP (built on Caddy).
The starting point of the application is the \texttt{public/index.php} file where all requests are directed.
From here a Laravel application instance is retrieved from the \texttt{bootstrap/app.php} file.
The incoming request is next sent to the HTTP kernel where all necessary service providers are bootstrapped, and application level middleware is executed.
These include the database, queue, validation and routing components among many.
Following this the router will dispatch the incoming request to route specific middleware, after which a route or controller will handle generating a response.
Finally, the response will be sent out through the HTTP kernel and application instance \cite{bean2015laravel}.

A diagram of the complete request lifecycle and MVC can be seen on Figure \ref{fig:lifecycle}

\begin{figure}[H]
  \centering
  \includegraphics[width=\linewidth]{./figures/lifecycle.png}
  \caption{Laravel request lifecycle}
  \label{fig:lifecycle}
\end{figure}

\subsection{Ecosystem}

Laravel provides many built in functionalities for use, such as mailing, testing, queues, notifications, storage, authentication and authorization among many.
All of these capabilities can be extended, or new ones can be added through the rich package ecosystem surrounding the framework \cite{laravel-docs}.
The creators of Laravel also maintain numerous first party packages and there are also big package maintainers, most notably Spatie \cite{spatie} whose team maintain many packages and tools for the framework and FilamentPHP, which is a plugin extendable admin panel and UI (user interface) framework \cite{filament}.