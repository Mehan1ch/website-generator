\section{Software as a Service}

In order to develop a software as a service application suite we first have to explore what a SaaS entails.
To do this, an overview and understanding of cloud computing and its related terminologies is necessary.
ISO defines cloud computing as a
``paradigm for enabling network access to a scalable and elastic pool of shareable physical or virtual resources with self-service provisioning and administration on-demand \cite{ISO/IEC22123-1}.''
The National Institute of Standards and Technology (NIST) defines five essential characteristics, four deployment models and three service models for the cloud \cite{mell2011nist}.

The essential characteristics are the following:
\begin{itemize}
  \item On demand self-service: resources can be provisioned without additional human assistance.
  \item Broad network access: users can access capabilities through standardized methods over the network.
  \item Resource pooling: the provider manages their cloud resources in such a way that it can serve multiple customers flexibly.
  \item Rapid elasticity: capabilities are scalable and re-assignable on demand.
  \item Measured service: the cloud monitors resource usage, and it is controlled and automatically optimized.
\end{itemize}

The aforementioned deployment models are:
\begin{itemize}
  \item Private cloud: a single organizational entity uses the cloud.
  \item Community cloud: a community composed of multiple organizations or users uses the infrastructure.
  \item Public cloud: the cloud is open for use to the public.
  \item Hybrid cloud: the cloud is a mix of the models mentioned before.
\end{itemize}

All the above deployment models may be either on- or off-premise and additionally owned or managed by the organization(s) using them, third parties or some combination of the former two.

Most importantly for the topic, the three service models are:
\begin{itemize}
  \item Infrastructure as a Service (IaaS): the provided resources are fundamental to computing, such as network, storage, or processing power.
  \item Platform as a Service (PaaS): the provided resources allow the customer to deploy applications to the cloud provider's platform without configuring the underlying computing infrastructure.
  \item Software as a Service (SaaS): ``The capability provided to the consumer is to use the provider's applications running on a cloud infrastructure. The applications are accessible from
    various client devices through either a thin client interface, such as a web browser (e.g.,
    web-based email), or a program interface. The consumer does not manage or control the
    underlying cloud infrastructure including network, servers, operating systems, storage, or
    even individual application capabilities, with the possible exception of limited user-specific application configuration settings \cite{mell2011nist}.''
\end{itemize}

There are many cloud providers available for use, most notably Amazon Web Services (AWS), Microsoft's Azure, and Google Cloud Platform (GCP), which boast numerous amounts of services belonging to the models described previously.

\subsection{Aspects and challenges of SaaS}

After establishing a definition of SaaS and its related terminologies, lets dive into the different aspects and challenges of developing, maintaining, and operating one.

``Software as a service (SaaS) is a software delivery model in which
third-party providers provide software as a service rather than as a
product over the Internet for multiple users \cite{aleem2017architecture}.''
SaaS applications hence are also web applications, and are built with web technologies and frameworks.
However, not all web applications are SaaS due to the aforementioned delivery model.
In a way SaaS is a specific subset of web applications.

There is not a fixed defined set of characteristics that a SaaS has to meet in order to be considered one, however there are some commonly recurring ones that most take into account, such as multi-tenancy, customization, and scalability \cite{espadas2008application}.

Most SaaS applications support some form of multi-tenancy.
Multi-tenancy is the ability to share the same (and single) hosted instance of the application between multiple tenants, instead of deploying the application separately for each tenant.
A tenant is a user or group of users (e.g., an organization) with the same share of the service \cite{krebs2012architectural}.
Commonly, people refer to multi-tenancy as the ability to have multiple organizations use the service with their own share and users, however, as we can see, this is not a strict requirement.
There are many ways to achieve multi-tenancy with different amounts of complexity in concern areas like architecture, scaling, and security.
The methods usually differ by how separated each tenant's share is on a database, data model and infrastructure level \cite{tsai_bai_huang_2014}.

This goes hand-in hand with the next common characteristic, which is customization. SaaS applications must support a level of customization to meet all tenants' needs. This allows tenants to adapt the service more seamlessly into their workflows, processes, and requirements. The ability to change aspects also reduces repetitive tasks between users of the same tenants.
Customization can range from interface level differences to full on tailoring of the tenant's share of the application down to the platform or even infrastructure level.
Each added aspect of personalization, however, increases complexity, due to which a balance must be struck between customization and standardization \cite{aslam2023benefits}.

Scalability is the ability of the service to handle increasing workloads and usage needs.
Generally there are two methods of scaling: vertical scaling (also known as scale-up) and horizontal scaling (also known as scale-out).
Vertical scaling involves giving more resources to the machine running the service, while horizontal scaling means distributing the application on multiple machines.
In a cloud environment, consumers rely on a scale-out solution more by default, due to the available amount of resources, and also due to the fact that increasing a machine's resources is only feasible up to a level.
However, most complex production systems usually combine the two forms in some way.
Not only that, SaaS platforms commonly employ either a two-level or a generalized K-level scalability structure where each level is a separate dimension of the service capable of scaling independently.
The platforms also make scaling solutions tenant-aware so that each tenant gets scaled for their respective needs \cite{scalable}.

Of course, plenty more aspects could be reviewed and explored regarding SaaS applications, however, the previously mentioned concerns and areas are sufficient to get a general understanding required for the topic.

Nowadays, many popular applications operate using a SaaS model such as Microsoft's Office suite, Slack, or Notion.
In regard to the thesis topic, there are also many similar existing solutions such as Squarespace, Wix, GoDaddy, or Shopify.
However, they almost always require a subscription or one time payment upfront, and/or only offer a limited trial.
They also frequently have a steep learning curve due to their complexity, and specialize in one given business area.