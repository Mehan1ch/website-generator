\section{Kubernetes}

In this section I'm going to provide an overview of Kubernetes and its minimal subset of features required to understand this thesis, as Kubernetes is a vast and complex topic by itself. I'm assuming a basic understanding and familiarity of Docker, containerization, and virtualization from the reader.

``Kubernetes is an open source container orchestration system. It
manages containerized applications across multiple hosts for
deploying, monitoring, and scaling containers. Originally created
by Google, in March of 2016 it was donated to the Cloud Native
Computing Foundation (CNCF) \cite{kubernetes2019kubernetes}.''

Kubernetes has multiple distributions, can be installed standalone, or used through the services of popular cloud platform providers.
It is used declaratively to configure the desired state of a system or an application.
Most often users interact with Kubernetes through its command line interface (CLI) with the \texttt{kubectl} command.
This interacts with the underlying API of Kubernetes.

Pods are either a single or a group of containers with shared network and storage resources, and are the smallest configurable and manageable deployment unit in the system.
A Pod can have any number of Init Containers.
They must run successfully and in a set order, and can be used to do setup tasks for their respective Pod.
Replicasets maintain a set number of replicas of the same pod, while Deployments provide a declarative way to update Replicasets or Pods.
Finally, a Service can be used to expose a network application that is running a set of pods behind a single outward facing endpoint \cite{luksa2017kubernetes}.

\subsection{Ingress, Gateway API}

Since Pods can be short-lived, often recreated, and have multiple replicas, they need a specialized service to access HTTP and HTTPS (HyperText Transfer Protocol Secure) routes from outside the cluster.
This is provided by the Kubernetes Ingress or Gateway API.

``Ingress exposes HTTP and HTTPS routes from outside the cluster to services within the cluster. Traffic routing is controlled by rules defined on the Ingress resource \cite{ingress_2025}.''

The Gateway API is, in a way, a successor to the Kubernetes Ingress, and it provides a more modern and scalable solution to route and traffic management \cite{Kanuru_Gurugubelli_2025}.

Both the Ingress and Gateway API need a controller or provider to function. One such industry standard is Traefik, which can serve both functionalities, and is an open source application proxy \cite{sharma2020traefik}.

A small illustration of the above concepts can be seen in Figure \ref{fig:traefik}.

\begin{figure}[H]
  \centering
  \includegraphics[width=\linewidth]{./figures/traefik.png}
  \caption{Traefik Gateway API}
  \label{fig:traefik}
\end{figure}

\subsection{Client Libraries}

Kubernetes provides official client libraries for several programming languages, including Go, Python, Java, JavaScript, and .NET, enabling programmatic interaction with the Kubernetes API.
Additional community maintained libraries are also available for most common programming languages and environments.

These libraries enable developers to communicate with the cluster through code instead of calling the REST API directly.
They provide typing support for requests, responses, and resources as well.
Moreover, they handle common tasks, which include authentication against the cluster, using and discovering service accounts, and understanding and parsing the \texttt{kubeconfig} to read addresses and credentials \cite{client_libraries_2025}.