\section{Authentication, dashboard, and account routes}

In this block, I will mainly be presenting the authentication and account related pages that I created on the frontend.
To begin with, a use case diagram in Figure \ref{fig:auth-use-case} shows in detail the authentication capabilities.

\begin{figure}[H]
  \centering
  \includegraphics[width=\linewidth]{./figures/auth-use-case.png}
  \caption{Authentication use cases}
  \label{fig:auth-use-case}
\end{figure}

Guests or unauthenticated users can log in, register, and ask for a forgot password link, which redirects them to a form allowing them to reset their password.
Authenticated users can verify their emails, which they automatically get an email for after registration, or can request a new one on the account page, which I will talk about later.
Obviously, if the user clicks on the verification link in their email while being unauthenticated they will first be redirected to the login page.

When a user first visits the frontend, a simple index page greets them with directions to login, or register, or go to the dashboard if they are already logged in.
The index page as shown for an unauthenticated user can be seen in Figure \ref{fig:index-page}.

\begin{figure}[H]
  \centering
  \includegraphics[width=\linewidth]{./figures/index-page.png}
  \caption{Index page for an unauthenticated user}
  \label{fig:index-page}
\end{figure}

From the landing page, the user can directly log in and/or register.
As stated previously, I used React Hook Form with Zod for creating all my forms.
This ensures that there is proper client side validation, and can be used to display meaningful error messages to the user.
All forms also employ a loading state to prevent multiple submissions while a previous request is running.
I also display success and error responses using the aforementioned toast notifications in the bottom right corner of the screen.
The registration page with an example for validation errors can be seen in Figure \ref{fig:register}.

\begin{figure}[H]
  \centering
  \includegraphics[width=\linewidth]{./figures/register.png}
  \caption{Register page with validation errors}
  \label{fig:register}
\end{figure}

The rest of the forms look almost identical to the registration one, so I won't show each of them one-by-one unnecessarily.
One thing worth mentioning is that both the registration and forgot password links send out emails.
Such an example email after registration captured by Mailpit can be seen in Figure \ref{fig:email}. On the sidebar the incoming password reset links can be seen as well.

\begin{figure}[H]
  \centering
  \includegraphics[width=\linewidth]{./figures/email.png}
  \caption{Registration email seen in Mailpit}
  \label{fig:email}
\end{figure}

After logging in, the user can access the main application routes grouped by the \texttt{\_app} folder.
They are all surrounded with a layout component that has two main responsibilities.
First, check if the user has successfully logged in, if not, redirect them to the login route.
Secondly, this is where the title callback I mentioned in Section \ref{section:frontend-setup} is used.
The layout component collects the result of all title callbacks from the router state, which include the title names for all rendered child routes.
Afterwards, it dynamically builds the breadcrumb trail displayed to the user.

The first such route is the dashboard, which displays a few statistics and quick actions.
Figure \ref{fig:dashboard} showcases it in detail.

\begin{figure}[H]
  \centering
  \includegraphics[width=\linewidth]{./figures/dashboard.png}
  \caption{The dashboard page}
  \label{fig:dashboard}
\end{figure}

Next, let me provide and overview of account settings, which can be accessed via a pop-over by clicking on the user section of the sidebar.
Another use case diagram showcasing the account-related options can be seen in Figure \ref{fig:account-use-case}.

\begin{figure}[H]
  \centering
  \includegraphics[width=\linewidth]{./figures/account-use-case.png}
  \caption{Account settings use cases}
  \label{fig:account-use-case}
\end{figure}

The user is able to switch the application between light, dark, and system themes.
They can also update their profile information, which includes changing their display name and email.
Changing the email automatically sends a verification email to the new address.
They can upload an avatar via a drag and drop field or the system file picker, which can be previewed before saving.
A tab, which shows the verification status is available, and if unverified, a new link can be sent from here.
Passwords can be changed using a form that shows the new password's strength using simple metrics like length and contained special characters.
After changing it, the user will be logged out automatically and prompted to log in again.
Lastly, the account can be deleted if the user wishes so.
This will first show a confirmation dialog to avoid accidental deletion.

Two pictures comparing light and dark modes can be seen in Figure \ref{fig:theming}.
The light mode picture on Subfigure \ref{subfig:light-avatar} showcases the avatar upload tab, while the dark mode picture on Subfigure \ref{subfig:dark-password} presents the password change form.
The light mode picture also displays how to get to the account settings page.

\begin{figure}[H]
  \centering
  \begin{subfigure}[b]{0.49\textwidth}
    \centering
    \includegraphics[width=\textwidth]{./figures/light-avatar.png}
    \caption{Light mode avatar upload}
    \label{subfig:light-avatar}
  \end{subfigure}
  \hfill
  \begin{subfigure}[b]{0.49\textwidth}
    \centering
    \includegraphics[width=\textwidth]{./figures/dark-password.png}
    \caption{Dark mode password change}
    \label{subfig:dark-password}
  \end{subfigure}
  \caption{Dark and light modes}
  \label{fig:theming}
\end{figure}