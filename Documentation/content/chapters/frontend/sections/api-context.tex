\section{API client and context providers}
\label{section:api-context}

After the setup process, I will move on to describing the features mainly in the \texttt{context}, \texttt{providers}, \texttt{hooks}, and \texttt{lib} folders.

Generating the OpenAPI schema of the backend has the great benefit of allowing me to take advantage of the many available libraries specializing in API client code generation.
I investigated a couple of existing solutions, namely Orval, Kubb, and OpenAPI-TS.
Orval and Kubb both feature an extendible approach where you can pick what you want to use.
However, they both proved to be way more complicated than what I needed, so I opted to use OpenAPI-TS.
OpenAPI-TS provides many smaller libraries that provide client-side code generation from OpenAPI schemas.
For my use case OpenAPI-React-Query was the one I needed and ended up using.
It is a tiny, around 1kb wrapper around TanStack Query and incorporates two other libraries by the same author, OpenAPI-Fetch and OpenAPI-TypeScript.
It validates the available API routes inside the query calls, checks parameters and bodies, and can be used to generate TypeScript types for the requests and responses.

Next I generated the API client with the library in the \texttt{lib/api} folder and also customized the fetch client used by it.
Namely, I created a custom \texttt{APIError} class that handles error messages from the backend, configured the base URL from environment variables and the default headers.
Additionally, since the backend uses CORS and CSRF protection it requires requests to have a valid XRSF token.
This can be obtained by first making a request to the \texttt{/sanctum/csrf-cookie} endpoint.
This was also configured in the fetch client to be handled automatically and re-fetch when the token expires.

Moving on, I implemented the custom theme and authentication providers.
I was guided by the ShadCN documentation \cite{shadcn_2025} on how to configure light, dark, and system themes using TailwindCSS's functionality.
The provider stores the selected theme in and recovers it from the browser's local storage on subsequent visits.
Toggling the themes are possible from the account page, and the app defaults to the system theme.
Accessing the theme in other components is possible through a custom \texttt{useTheme} hook.

Afterwards I created the authentication provider for auth state management.
It provides through the \texttt{useAuth} hook methods for logging in, registering, logging out, re-fetching user context and deleting the account.
All these methods use the API client wrapping TanStack Query's mutations and query functions.
The context encompasses two React states, one a boolean for quick checks, and the other containing the logged-in user's information.
I also store the state in the browser's local storage and recovered upon mounting by an \texttt{useEffect} call so that the user remains logged in on revisits.