\section{Schemas, sites, and pages}

Next up, let me talk about creating and managing schemas.
As with the previous features, I present a use-case diagram of them in Figure \ref{fig:schema-use-cases}.

\begin{figure}[H]
  \centering
  \includegraphics[width=\linewidth]{./figures/schema-use-cases.png}
  \caption{Schema use cases}
  \label{fig:schema-use-cases}
\end{figure}

Schemas can be used as templates for pages, which I will mention the details of later.
Most of the use cases include basic CRUD operations.
Users, however, have to have a verified email to access schemas (sites and pages too), and also they can only view them. They also only see published schemas.
Only administrators are capable of editing, designing, and publishing schemas for users to reuse.
I used similar page designs for related features across schemas, pages, and sites, so I won't show each page related to the schemas here.
In Figure \ref{fig:schema-list} a list view of schemas can be seen.

\begin{figure}[H]
  \centering
  \includegraphics[width=\linewidth]{./figures/schema-list.png}
  \caption{Card list of schemas}
  \label{fig:schema-list}
\end{figure}

At the bottom of the screen, the pagination component is visible, and the hovered first card has a shadow and shading applied.
The detailed view of schemas is almost identical to the pages.
For an idea on how that looks, see Figure \ref{fig:page-detail}.
The schema edit and create page is basically the same, and it is a simple form that can be viewed in Figure \ref{fig:schema-edit}.

\begin{figure}[H]
  \centering
  \includegraphics[width=\linewidth]{./figures/schema-edit.png}
  \caption{Schema edit form}
  \label{fig:schema-edit}
\end{figure}

The designer for schemas is the same as the one for the pages and is the main topic of Section \ref{section:editor}.

The next resource on the frontend is the sites.
Once again, a use case diagram of site related features is visible in Figure \ref{fig:site-use-cases}.

\begin{figure}[H]
  \centering
  \includegraphics[width=\linewidth]{./figures/site-use-cases.png}
  \caption{Site use cases}
  \label{fig:site-use-cases}
\end{figure}

Sites encompass their pages. They have a two tabbed detail view, one showing general information, while the other shows their pages.
It also contains controls for deploying the application if it has not been published yet.
If it has, it is possible to restart, update and delete it from the dropdown.
A confirmation dialog appears in all mentioned cases.
The detail view of the site appears in Figure \ref{fig:site-detail}.
Subfigure \ref{subfig:site-overview} shows the general information, deployment controls, buttons that take the user to the edit form and designer pages, and a delete button for deleting the site and all its resources.
A toast notification indicating a successful restart is also visible.
On the other hand, Subfigure \ref{subfig:site-pages} shows the pages of the site where their detail view can be visited by clicking on a card, or a new one can be created with the provided button.

\begin{figure}[H]
  \centering
  \begin{subfigure}[b]{0.49\textwidth}
    \centering
    \includegraphics[width=\textwidth]{./figures/site-restart.png}
    \caption{Overview tab}
    \label{subfig:site-overview}
  \end{subfigure}
  \hfill
  \begin{subfigure}[b]{0.49\textwidth}
    \centering
    \includegraphics[width=\textwidth]{./figures/site-pages.png}
    \caption{Pages tab}
    \label{subfig:site-pages}
  \end{subfigure}
  \caption{Site detail view}
  \label{fig:site-detail}
\end{figure}

Regarding the deployment options, this and the schema publish is the only scenario in the frontend where I have to use manual query invalidation to update the displayed data in time.
This is because we stay on the same view, instead of navigating away, which would normally trigger SWR's re-fetch mechanism.
All three resources I talk about in this section also have an empty component that is displayed if the returned API answer is an empty collection.
Figure \ref{fig:site-empty} displays the one belonging to sites.

\begin{figure}[H]
  \centering
  \includegraphics[width=\linewidth]{./figures/site-empty.png}
  \caption{Site's empty component}
  \label{fig:site-empty}
\end{figure}

Finally, let me talk about the pages. As with the previous two cases, Figure \ref{fig:page-use-cases} shows a use case diagram for existing page related actions.

\begin{figure}[H]
  \centering
  \includegraphics[width=\linewidth]{./figures/page-use-cases.png}
  \caption{Page use cases}
  \label{fig:page-use-cases}
\end{figure}

The editor related actions, such as generating the HTML content, play core parts in saving the page to the backend.
Most other functions are identical to schemas and sites.
A crucial difference is the creation form.
Since pages can be based off of schemas, the creation form needs to be able to provide the ability to select a schema as the base content for a page.
Later, this of course can be edited by the user without it having any effect on the original schema.
The form basically copies its content into the page.
To solve this problem, I created a multistep form using the Stepperize React library, which has built-in support for the ShadCN UI components \cite{stepperize_2025}.
Figure \ref{fig:page-create} showcases this form and the schema selection step. Schemas can be previewed here via a dialog.

\begin{figure}[H]
  \centering
  \includegraphics[width=\linewidth]{./figures/page-create.png}
  \caption{Page creation multistep form}
  \label{fig:page-create}
\end{figure}

At the end of the form I show the user a review tab, and the process can also be reset here if needed.
The page edit form is similar to the schema edit form in Figure \ref{fig:schema-edit}.
The detail view of the pages also contains a preview of the content for easier identification.
Figure \ref{fig:page-detail} presents this functionality.

\begin{figure}[H]
  \centering
  \includegraphics[width=\linewidth]{./figures/page-detail.png}
  \caption{Page details with content preview}
  \label{fig:page-detail}
\end{figure}