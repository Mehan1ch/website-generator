\pagenumbering{roman}
\setcounter{page}{1}
\selectenglish

%----------------------------------------------------------------------------
% Abstract in English
%----------------------------------------------------------------------------
\chapter*{Abstract}\addcontentsline{toc}{chapter}{Abstract}

\selectthesislanguage

\newcounter{romanPage}
\setcounter{romanPage}{\value{page}}
\stepcounter{romanPage}

This master's thesis presents the development of a template-driven website generator SaaS platform. Motivation stems from the observation that current website builders and e-commerce platforms, while powerful, are often feature-heavy, costly, and impose a steep learning curve.
For users who only require basic informational sites (e.g., resumes, portfolios, local business pages), the complexity and recurring subscription costs of mainstream SaaS offerings present a practical barrier.

The implemented solution composes a Laravel backend for data and user management, a React-based frontend with a streamlined component based editor, and a lightweight Node.js proxy to programmatically deploy user sites to a containerized environment.
Sites are modelled as collections of static pages generated from templates and user content, the platform automates publishing, routing, and basic security considerations.
The implementation emphasizes modularity, maintainability, and testability.

Results include a complete prototype, a documented end-to-end usage example, and comprehensive testing validating core user flows.
The prototype demonstrates that non-technical users can create and publish a static informational site within minutes, without complex configuration and technical knowledge required.

Production considerations are also discussed, covering containerization, CDN-backed asset delivery, managed Kubernetes, CI/CD workflows, secrets and backup strategies, observability, tenant isolation, and autoscaling.

In conclusion, the implemented SaaS can effectively serve many basic website needs.
Future work includes additional components and customizability, revision history and rollback, stronger multi-tenancy, automated backups, richer observability, and evaluation of production-scale resilience and performance.

\newpage
\selecthungarian

%----------------------------------------------------------------------------
% Abstract in Hungarian
%----------------------------------------------------------------------------
\chapter*{Kivonat}\addcontentsline{toc}{chapter}{Kivonat}

Jelen diplomamunka egy sablonvezérelt weboldal-generáló SaaS platform fejlesztését mutatja be.
A motiváció abból ered, hogy a jelenlegi weboldal készítők és elektronikus kereskedelmi platformok, bár funkcióban gazdagok, gyakran túl sok funkciót kínálnak, költségesek, és meredek tanulási görbével járnak.
Azoknak a felhasználóknak, akik csak egyszerű információs oldalakat igényelnek (például önéletrajz, portfólió vagy helyi vállalkozás bemutató oldala), az eljterjedt SaaS szolgáltatások összetettsége és ismétlődő előfizetési költségei gyakorlati akadályt jelentenek.

A megvalósított megoldás egy Laravel alapú backendből a felhasználó és adatkezeléshez, egy React alapú kliensből egy egyszerűsített komponensalapú szerkesztővel, valamint egy könnyű Node.js proxyból áll, amely programatikusan telepíti a felhasználói oldalakat a konténerizált környezetbe.
A weboldalakat sablonokból és felhasználói tartalomból generált statikus oldalak gyűjteményeként modelleztem, a platform automatizálja a publikálást, a routingot és az alapvető biztonsági beállításokat.
A megvalósítás során a modularitásra, karbantarthatóságra és tesztelhetőségre helyeztem a hangsúlyt.

Eredményként bemutatok egy működő prototípust, egy részletes és teljes példát a használatra, valamint átfogó teljeskörű teszteket, amelyek igazolják az alapvető felhasználói folyamatok helyes működését.
A példa szemlélteti, hogy nem szakmai felhasználók percek alatt létrehozhatnak és publikálhatnak egy statikus információs oldalt, bonyolult konfiguráció és mély technikai ismeretek nélkül.

A dolgozat tárgyalja az üzemeltetési megfontolásokat is: konténerizációt, CDN-támogatott tartalomszolgáltatást, menedzselt Kubernetes megoldásokat, CI/CD munkafolyamatokat, titkok és mentések kezelését, megfigyelhetőséget, izolációt és automatikus skálázás szempontjait.

Összefoglalva, a megvalósított, szűkebb funkcionalitású SaaS hatékonyan képes kielégíteni számos alapvető weboldal igényeit. További fejlesztések közé tartozhatnak további komponensek és jobb testreszabhatóság, verziókezelés és visszaállítás, erősebb multi-tenancy, automatikus mentések, kiterjedtebb megfigyelhetőség, valamint az éles környezetben való üzemeltetés rendelkezésre állásának és teljesítmények vizsgálata.

\selectenglish